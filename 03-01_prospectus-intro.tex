\documentclass[12pt,]{article}
\usepackage[left=1in,top=1in,right=1in,bottom=1in]{geometry}
\newcommand*{\authorfont}{\fontfamily{phv}\selectfont}
\usepackage[]{mathpazo}


  \usepackage[T1]{fontenc}
  \usepackage[utf8]{inputenc}




\usepackage{abstract}
\renewcommand{\abstractname}{}    % clear the title
\renewcommand{\absnamepos}{empty} % originally center

\renewenvironment{abstract}
 {{%
    \setlength{\leftmargin}{0mm}
    \setlength{\rightmargin}{\leftmargin}%
  }%
  \relax}
 {\endlist}

\makeatletter
\def\@maketitle{%
  \newpage
%  \null
%  \vskip 2em%
%  \begin{center}%
  \let \footnote \thanks
    {\fontsize{18}{20}\selectfont\raggedright  \setlength{\parindent}{0pt} \@title \par}%
}
%\fi
\makeatother




\setcounter{secnumdepth}{0}




\title{Rivalry, Collective Action, and Cooperation in International Development  }



\author{\Large Miles D. Williams\vspace{0.05in} \newline\normalsize\emph{University of Illinois at Urbana-Champaign}  }


\date{}

\usepackage{titlesec}

\titleformat*{\section}{\Large\bfseries}
\titleformat*{\subsection}{\large\itshape\bfseries}
\titleformat*{\subsubsection}{\normalsize\itshape}
\titleformat*{\paragraph}{\normalsize\itshape}
\titleformat*{\subparagraph}{\normalsize\itshape}


\usepackage{natbib}
\bibliographystyle{plainnat}
\usepackage[strings]{underscore} % protect underscores in most circumstances



\newtheorem{hypothesis}{Hypothesis}
\usepackage{setspace}


% set default figure placement to htbp
\makeatletter
\def\fps@figure{htbp}
\makeatother

\usepackage{amsmath}
\usepackage{dcolumn}

% move the hyperref stuff down here, after header-includes, to allow for - \usepackage{hyperref}

\makeatletter
\@ifpackageloaded{hyperref}{}{%
\ifxetex
  \PassOptionsToPackage{hyphens}{url}\usepackage[setpagesize=false, % page size defined by xetex
              unicode=false, % unicode breaks when used with xetex
              xetex]{hyperref}
\else
  \PassOptionsToPackage{hyphens}{url}\usepackage[draft,unicode=true]{hyperref}
\fi
}

\@ifpackageloaded{color}{
    \PassOptionsToPackage{usenames,dvipsnames}{color}
}{%
    \usepackage[usenames,dvipsnames]{color}
}
\makeatother
\hypersetup{breaklinks=true,
            bookmarks=true,
            pdfauthor={Miles D. Williams (University of Illinois at Urbana-Champaign)},
             pdfkeywords = {},  
            pdftitle={Rivalry, Collective Action, and Cooperation in International Development},
            colorlinks=true,
            citecolor=blue,
            urlcolor=blue,
            linkcolor=magenta,
            pdfborder={0 0 0}}
\urlstyle{same}  % don't use monospace font for urls

% Add an option for endnotes. -----


% add tightlist ----------
\providecommand{\tightlist}{%
\setlength{\itemsep}{0pt}\setlength{\parskip}{0pt}}

% add some other packages ----------

% \usepackage{multicol}
% This should regulate where figures float
% See: https://tex.stackexchange.com/questions/2275/keeping-tables-figures-close-to-where-they-are-mentioned
\usepackage[section]{placeins}


\begin{document}
	
% \pagenumbering{arabic}% resets `page` counter to 1 
%
% \maketitle

{% \usefont{T1}{pnc}{m}{n}
\setlength{\parindent}{0pt}
\thispagestyle{plain}
{\fontsize{18}{20}\selectfont\raggedright 
\maketitle  % title \par  

}

{
   \vskip 13.5pt\relax \normalsize\fontsize{11}{12} 
\textbf{\authorfont Miles D. Williams} \hskip 15pt \emph{\small University of Illinois at Urbana-Champaign}   

}

}






\vskip -8.5pt


 % removetitleabstract

\noindent \doublespacing 

\hypertarget{introduction}{%
\section{Introduction}\label{introduction}}

From the Bretton Woods conference in 1944, to national security debates
in the 21\(\text{st}\) century, leaders of wealthy countries have long
recognized that economic development and political stability go hand in
hand. However, notwithstanding U.S. concerns about European economic
recovery after World War II, in decades past, most leaders have not
perceived underdevelopment, and the relatively localized security
problems it generates, as major international threats. During the Cold
War, the ideological and geopolitical standoff between the West and the
Soviet Union dominated the objectives of foreign policy. Tools like
foreign aid, though ostensibly intended to promote development in the
neediest countries, often served other diplomatic ends---like supporting
autocratic regimes loyal to the West. However, after the Cold War, and
especially since the onset of the War on Terror, promoting international
development has taken center stage in high level foreign policy debates
within and between the world's leading industrialized states.

In a globalized world, problems in developing countries have
transnational consequences. Convinced that underdeveloped parts of the
world serve as incubators of extremism and insurgency; as drivers of
illicit drug flows, human trafficking, mass migration, refugees, and
potential pandemics; leaders of wealthy countries view economic aid as
an indispensable tool in the promotion of international stability,
security, and prosperity. In an interconnected world, international
development and international security are inseparable.

While the above foreign policy objectives have been well studied, what
remains unclear is how industrialized countries' simultaneous efforts to
promote international development either create opportunities for
international collaboration in aid policy, or else spiral into
free-riding or strategic competition for diplomatic influence. As common
interest in addressing the root causes of discontent, instability, and
violence in developing countries has emerged among industrialized
states, this has generated repeated calls for cooperation among aid
donors. Collaborative arrangements include transparency and data-sharing
initiatives supported by Development Assistance Committee (DAC) members
of the Organization for Economic Co-operation and Development (OECD),
delegation of aid funds to multilateral agencies like the World Bank and
various regional development banks, the creation of joint assistance
strategies (JAS) among donor countries, and sector-wide approaches
(SWAPs) among donors to pool resources in supporting sector-specific aid
programs (Lawson 2013).

\emph{Have efforts to promote cooperation been successful?} When
\emph{and} where \emph{do foreign aid donating countries succumb to
myopic incentives to free-ride, exploiting the aid expenditures of
peers, or else compete with the aid of geostrategic and market rivals?}

These are not new questions, but firm answers remain allusive. On the
one hand, collaboration is made difficult by the fact that engaging with
a low- or middle-income state to promote development is not always a
politically neutral act. Leaders of donor states often expect to receive
political and economic concessions in exchange for providing development
resources to recipient countries. These may include market access,
political favors, military alliances, and so
on.\footnote{This is not to mention possible domestic gains that stem from policymakers' ability to take credit for successful aid projects.}
Such goals may lead to competition among industrialized states if these
benefits come at the expense of
others.\footnote{For example, in an anonymous interview a German aid official once complained that Germany's position as only the seventh largest bilateral ODA donor to Bangladesh "diluted" Germany's influence in this country (Steinwand 2015).}
Such competition can be inefficient, since it leads donors to give more
aid than is minimally necessary to receive their desired concessions
(see Bueno de Mesquita and Smith 2016), while the pursuit of
nondevelopment goals can detract from the effectiveness of aid in
promoting economic development (see Bearce and Tirone 2010).

Aside from these strategic goals, common interest in promoting
development to mitigate diffuse spillovers of developing country
problems creates a classic public goods problem, generating incentives
to free-ride on fellow donors. This can result in Pareto inefficient
supply of aid funding by industrialized countries when donors pursue
goals that benefit the entire international community. Free-riding can
even be a concern in promoting strategic aims when donors have
complementary geostrategic and commercial interests.

To cooperate successfully, a clear understanding of the factors that
facilitate ease of cooperation, and of the sources of cooperation
failure, is necessary. Unfortunately, though a small but growing cadre
of scholars has applied a diversity of theoretical and empirical
approaches to-date to shed light on these issues, conclusive answers
remain elusive. Even more, despite the relevance of collaboration in
international development, the issue comprises only a narrow subset of
development policy research.

In this dissertation project, I intend to fill this gap in the
literature. Further, moving beyond previous efforts to analyze
cooperation (failure) in international development, I propose starting
from a more rigorous theoretical foundation than has been considered
to-date. In so doing, I will be able to more clearly identify the
conditions under which free-riding, competition, and cooperation are
likely to occur among industrialized countries in promoting
international development. The predictions my theoretical approach
generates, and the results from the empirical analyses I undertake to
test these predictions, will represent a novel contribution to the
literature on the political economy of international development, which
has hitherto lacked a clear and well-supported statement on strategic
and cooperative action among foreign aid donors.

In the sections that follow in this dissertation prospectus, I first
provide motivation for my research. In particular, I highlight the
relevance of international development for contemporary international
politics in light of increased perception by world leaders of the threat
posed by poverty and political instability in developing countries. I
further situate this contemporary concern in a broader context that
dates back to the aftermath of World War II.

Having provided necessary background details, I subsequently lay out a
game theoretic model of a multi-public goods economy that I argue best
captures the strategic dynamics that states face in their efforts to
promote international development. I then situate my theoretical
argument in reference to previous studies that address donor
interactions.

After laying out the theory, I subsequently test a subset of hypotheses
with dyadic panel data on the bilateral aid commitments of DAC members
of the OECD. I also briefly discuss novel measures that I development to
test my theoretical argument. As the results will show, patterns in aid
allocation that I identify are consistent with my theoretical argument;
though, much more work remains to be done.

I finally provide an overall outline of the broader dissertation
project. I discuss extensions to the theoretical model that I will
address in the dissertation, and additional analyses I will undertake in
a series of empirical chapters. I further provide a rough timeline for
completing the dissertation.

\newpage
\Large
\setlength\parindent{-5pt}

\textbf{References} \normalsize \doublespacing

Barthel, Fabian, Eric Neumayer, Peter Nunnenkamp, and Pablo Selaya.
2014. ``Competition for Export Markets and the Allocation of Foreign
Aid: The Role of Spatial Dependence among Donor Countries.'' \emph{World
Development} 64: 350-65.

Bearce, David H. and Daniel C. Tirone. 2010. ``Foreign Aid Effectiveness
and the Strategic Goals of Donor Governments.'' \emph{Journal of
Politics} 72(3): 837-51.

Bermeo, Sarah B. 2016. ``Aid is Not Oil: Donor Utility, Heterogeneous
Aid, and the Aid-Democratization Relationship.'' \emph{International
Organization} 70(1): 1-32.

Bermeo, Sarah B. 2018. \emph{Targeted Development: Industrialized
Country Strategy in a Globalizing World.} New York: Oxford University
Press.

Brautigam, Deborah A. and Stephen Knack. 2004. ``Foreign Aid,
Institutions, and Governance in Sub-Saharan Africa.'' \emph{Economic
Development and Cultural Change} 52: 255-85.

Bueno de Mesquita, Bruce and Alastair Smith. 2009. ``A Political Economy
of Aid.'' \emph{International Organization} 63(2): 309-40.

Bueno de Mesquita, Bruce and Alastair Smith. 2016. ``Competition and
Collaboration in Aid-for-Policy Deals.'' \emph{International Studies
Quarterly} 60(3): 413-426.

Chafer, Tony. 2005. ``Chirac and `La Francafrique': No Longer a Family
Affair.'' \emph{Modern and Contemporary France} 13(1): 7-23.

Cornes, Richard and Jun-Ichi Itaya. 2010. ``On the Private Provision of
Two or More Public Goods.'' \emph{Journal of Public Economic Theory}
12(2): 363-85.

Davies, Ronald B. and Stephan Klasen. 2019. ``Darlings and Orphans:
Interactions across Donors in International Aid.'' \emph{The
Scandinavian Journal of Economics} 121(1): 243-77.

Dudley, Leonard. 1979. ``Foreign Aid and the Theory of Alliances.''
\emph{The Review of Economics and Statistics} 61(4): 564-71.

Dunning, Thad. 2004. ``Conditioning the Effects of Aid: Cold War
Politics, Donor Credibility, and Democracy in Africa.''
\emph{International Organization} 58: 409-23.

Fleck, Robert K. and Christopher Kilby. 2010. ``Changing Aid Regimes?
U.S. Foreign Aid from the Cold War to the War on Terror.'' \emph{Journal
of Development Economics} 91: 185-97.

Friedman, Milton. 1958. ``Foreign Economic Aid: Means and Objectives.''
\emph{The Yale Review} 47(4): 500-16.

Frot, E. and J. Santiso. 2011. ``Herding in Aid Allocation.''
\emph{Kyklos} 64: 54-74.

Fuchs, Andreas, Peter Nunnenkamp, and Hannes Ohler. 2015. ``Why Donors
of Foreign Aid Do Not Coordinate: The Role of Competition for Export
Markets and Political Support.'' \emph{The World Economy} 38(2): 255-85.

Gaddis, John Lewis. 2005. \emph{Strategies of Containment: A Critical
Appraisal of American National Security Policy During the Cold War}.
Revised and expanded ed.~New York Oxford University Press.

Gilpin, Robert. 1987. \emph{The Political Economy of International
Relations}. Princeton: Princeton University Press.

Hayter, Theresa. 1966. \emph{French Aid}. London: Overseas Institute.

Heinrich, Tobias. 2013. ``When is Foreign Aid Selfish, When is it
Selfless?'' \emph{Journal of Politics} 75(2): 422-35.

Heinrich, Tobias, Yoshiharu Kobayashi, and Kristin A. Bryant. 2016.
``Public Opinion and Foreign Aid Cuts in Economic Crises.'' \emph{World
Development} 77: 66-79.

Holmstrom, Bengt. 1982. ``Moral Hazard in Teams.'' \emph{The Bell
Journal of Economics} 13: 324-40.

Jervis, Robert. 1992. ``A Usable Past for the Future.'' In \emph{The End
of the Cold War: Its Meaning and Implications}. ed.~M.J. Hogan. New
York: Cambridge University Press.

Kegley, Charles W. Jr.~1993. ``The Neoidealist Moment in International
Studies? Realist Myths and the New International Realities: ISA
Presidential Address March 27, 1993 Acapulco, Mexico.''
\emph{International Studies Quarterly} 37(2): 131-46.

Keohane, Robert O. and Joseph S. Nye. 1977. \emph{Power and
Interdependence: World Politics in Transition}. Boston: Little, Brown.

Kolodziej, Edward A. 1974. \emph{French Foreign Policy under de Gaulle
and Pompidou: The Politics of Grandeur.} Ithaca: Cornell University
Press.

Lancaster, Carol. 2008. \emph{George Bush's Foreign Aid: Transformation
or Chaos?} Washington, DC: Center for Global Development.

Lawson, Marian Leonardo. 2013. ``Foreign Aid: International Donor
Coordination of Development Assistance.'' \emph{Congressional Research
Service}.

Mallaby, Sebastian. 2005. \emph{The World's Banker: A Story of Failed
States, Financial Crises, and the Wealth and Poverty of Nations.} New
York: Penguin Press.

Mascarenhas, Raechelle and Todd Sandler. 2006. ``Do Donors Cooperatively
Fund Foreign Aid?'' \emph{Review of International Organizations} 1:
337-57.

Morgenthau, Hans. 1962. ``A Political Theory of Foreign Aid.'' \emph{The
American Political Science Review} 56(2): 301-9.

Mosley, Paul. 1985. ``The Political Economy of Foreign Aid: A Model of
the Market for a Public Good.'' \emph{Economic Development and Cultural
Change} 33(2): 373-93.

Nunnenkamp, Peter, Hannes Ohler, and Rainer Thiele. 2013. ``Donor
Coordination and Specialization: Did the Paris Declaration Make a
Difference?'' \emph{Review of World Economics} 149: 537-63.

Olson, Mancur and Richard Zeckhauser. 1966. ``An Economic Theory of
Alliances.'' \emph{Review of Economics and Statistics} 48(3): 266-79.

Pahre, Robert. 1995. ``Wider and Deeper: The Links between Expansion and
Integration in the European Union.'' In \emph{Towards a New Europe:
Stops and Starts in Regional Integration}. Eds. Gerald Schneider,
Patricia A. Weitsman, and Thomas Bernauer. Westport: Praeger.

Radelet, Steven. 2003. ``Bush and Foreign Aid.'' \emph{Foreign Affairs}
82(5): 104-17.

Rahman, Aminur and Yasuyuki Sawada. 2012. ``Can Donor Coordination Solve
the Aid Proliferation Problem?'' \emph{Economic Letters} 116(3): 609-12.

Schraeder, Peter J., Steven W. Hook, and Bruce Taylor. 1998.
``Clarifying the Foreign Aid Puzzle: A Comparison of American, Japanese,
French, and Swedish Aid Flows.'' \emph{World Politics} 50: 294-320.

Steinwand, Martin C. 2015. ``Compete or Coordinate? Aid Fragmentation
and Lead Donorship.'' \emph{International Organization} 69(spring):
443-72.

Terza, Joseph V., Anirban Basu, and Paul J. Rathouz. 2008. ``Two-Stage
Residual Inclusion Estimation: Addressing Endogeneity in Health
Econometric Modeling.'' \emph{Journal of Health Economics} 27(3):
531-43.

Woods, Ngaire. 2005. ``The Shifting Politics of Foreign Aid.''
\emph{International Affairs} 81(2): 393-409.

Woods, Randall B. 1997. \emph{The Marshall Plan: A Fifty Year
Perspective.} Lexington: George C. Marshall Foundation.

Wright, Joseph. 2009. ``How Foreign Aid Can Foster Democratization in
Authoritarian Regimes.'' \emph{American Journal of Political Science}
53(3): 552-71.





\newpage
\singlespacing 
\end{document}